\documentclass{article}
\usepackage[utf8]{inputenc}
\usepackage{amsmath}
\title{Ass 4}
\author{bhavita }
\date{May 2018}

\begin{document}

\maketitle
\section{Q.2}
It is the same as finding number of closed polygons formed from Kn i.e (n-1)! but here since there is no difference between clockwise and anticlockwise direction thus no.of hamiltonian cycles formed $=(n-1)!/2$ 

\section{Q.3}
\subsection{i}
The total subgraphs of Kn=$\sum_{m=0}^{n}{n\choose  m}2^{m\choose  2}  $
\subsection{ii}
An undirected graph has an Euler cycle if and only if it is connected
and has all vertices of even degree. Thus, we want to find Kn ,where each vertex will be of even degree.Given n vertices, Kn will have $n(n -1)/2 $ edges.If the total degree of Kn is divided by n vertices, the result will be even, thus the Kn having an Euler cycle. We also know the total degree of any graph is twice the number of edges thus total degree is $(n(n-1))$

Now we take the total degree $(n(n -1))$ and divide it by n vertices for any Kn
graph and the result is $(n -1)$ and this  is the degree of each vertex in any Kn graph.
Thus, for a Kn graph to have an Euler cycle, we want $(n - 1)$ to be an even value. But we already know in terms of complete graphs, if the number of n vertices is odd, we will have an even $(n -1)$ value and if n is even, we will have an odd $(n -1)$value.Thus if n is odd Kn will have euler cycle 

\section{Q.4}
\subsection{i}
The given number of vertices are 5.We would solve by taking possibilities of number edges:
E=0 is the min and max is 10 (suppose there are n vertices given,then for every vertex there is a possibility of attaching with n-1 vertices thus edges are n*(n-1) but as it is a simple graph(undirected)thus E=n*(n-1)/2)

       for E=0 -: no.of graphs= ${10\choose  0}$

for E=1 -: no.of graphs= ${10\choose  1}$

    .
    
    .
    
    .
    
    .
    
for E=10 -: no.of graphs= ${10\choose  10}$

Thus the total sum is $2^{10}$
\subsection{ii}
We would solve by taking cases-:

   

0 edge   =     1  graphs

1 edges  =     1   graphs

2   egdes =    2    graphs

3  egdes  =    3   graphs

4  edges =      2   graphs

5  edges =      2   graphs
\section{Q.5}
The sum of the degree of the graph can never be odd,it is always even  because every edge is made up of 2 vertices thus while counting the degree we take the edge twice i.e once for each vertex.
As the sum of the given degree sequence is 13(odd),thus this degree sequence is not possible.

\section{Q.7}
We would proceed through contradiction-:

Let us assume the graph to be disconnected that means there exists atleast two vertices let's say (n,m) that are not connected i.e $degree(m)+degree(n)<=(n-2)$ hence this conradicts the statement

\section{Q.8}
 Let path P be the longest path in graph G that starts at point A, which has the lowest degree of the graph. If A has degree higher than 2, then it will have some neighboring point W. If W was not on P, then A would have a neighbor that's not on P, and then P would not be the longest path, so unless A's other neighbor is on the path as well, this creates a contradiction. As a result, W has to be on P, otherwise we'd have the said contradiction, meaning that all of A's neighbors are on P. This creates a cycle, and since a cycle has length of at least $|V(G)| + 1$, this path will have length of at least k + 1
 
\end{document}
